
\documentclass[10pt]{article}
\usepackage[utf8]{inputenc}
\usepackage{float}
\usepackage{url}
\usepackage[portuguese]{babel}
\usepackage{graphicx}
\usepackage{microtype}
\usepackage[T1]{fontenc}

\title{IF681 -INTERFACES USUARIO-MAQUINA}
\author{ Geozedeque Guimarães}
\date{May 2019}

\usepackage{natbib}

\begin{document}

\maketitle

\section{Introdução}

\begin{figure}[h!]
\centering
\includegraphics[scale=0.5]{iron1.jpg}
\caption{Relação homem e interface}\cite{Imagem}
\label{fig:Relação homem e interface}
\end{figure}

 A interface faz parte do sistema computacional e determina como as pessoas operam e controlam o sistema. Quando uma interface é bem projetada, ela é compreensível, agradável e controlável. Os usuários se sentem satisfeitos e seguros ao realizar suas ações.\cite{Introd}
 
 Sendo assim, a disciplina de Interfaces Usuário-Máquina compreende o escopo de conhecer as necessidades de um público e idealizar uma solução para a necessidade encontrada, conversando com pessoas que conhecem e ou vivenciam o tema, para que a solução desenvolvida seja pertinente.\cite{Introd1}

\section{Objetivos}\cite{Objetivo}
    Após o curso, o discente deve ser capaz de:

\begin{itemize}
  \item Aprender a pesquisar sobre soluções já existentes para o dado problema;
  \item Compreender as necessidades do público alvo, buscando as informações diretamente com o mesmo;
  \item Ter conhecimento básico sobre organização e filtro de ideias em grupo;
  \item Desenvolver protótipos;
  \item Validar protótipos com especialistas ou clientes.
\end{itemize}

\section{Relevância da disciplina}

Interfaces usuário maquina, é uma área de estudo cuja importâcia é voltada para compreensão de todo processo de comunicação que ocorre entre o homem e um computador. Visando tornar esta interação mais simples por meio de interfaces “amigáveis”, de modo que todo usuário possa utilizar os recursos disponíveis em uma máquina de maneira fácil e eficiente.\cite{Importancia}

\section{Relação Interdisciplinar}

Durante o tempo de estudo da cadeira se cobrará do aluno habilidade em Programação como ferramenta para compreensão e desenvolvimento das atividades na respectiva cadeira de IF681 Intenterfaces usuário maquina.

\begin{center}
\begin{tabular}{|c|p{6cm}|}

\hline
Identificação da Cadeira & Sobre a Cadeira: \\ \hline
 IF669 Introdução à Programação &
 A disciplina de Introdução a Programação expõe conceitos e técnicas fundamentais de programação, com enfoque em linguagens orientadas a objetos e utilizando a linguagem Java.\cite{IF669}
 \\ \hline

\end{tabular}    
\end{center}

\bibliographystyle{plain}
\bibliography{ggs2}
\end{document}